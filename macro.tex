\documentclass[a4paper,11pt,titlepage,uplatex]{jsarticle}

\usepackage[dvipdfmx]{graphicx,xcolor}% ドライバ指定
\usepackage[top=30truemm,bottom=30truemm,left=25truemm,right=25truemm]{geometry} % 余白設定

\usepackage{here, subfig} % 画像
\graphicspath{{./pic/}}

\usepackage{docmute} % ファイル分割用

% 表関連のパッケージ
\usepackage{booktabs}
\usepackage{multirow}
\usepackage{longtable}
\usepackage{arydshln} % 表で破線を使うため
\usepackage{multicol}
% longtableをusepackageする場合は順番が重要らしいです。longtableとarydshlnの順番逆にしたらエラーはく(コンパイルはできるが…)

% 数式関連
\usepackage{amsmath,amsfonts,amssymb,mathtools,amsthm}

\usepackage{empheq} % 連立方程式をきれいに書いてくれる
\usepackage{setspace} % 行間を変えることができる。行列に用いれば良い
\usepackage{bm}
\usepackage{physics}
\usepackage[version=3]{mhchem}
\usepackage[separate-uncertainty]{siunitx}
%\usepackage{txfonts} % 行列のかっこの形に影響あり

\usepackage{enumitem} % enumium環境いじるため
\renewcommand{\labelenumi}{\theenumi.}
\renewcommand{\theenumi}{\Alph{enumi}}

% ------------ url関係
\usepackage{url}
\usepackage[dvipdfmx]{hyperref}
\hypersetup{
	colorlinks=true,
	citecolor=blue,
	linkcolor=red,
	urlcolor=blue
}
\usepackage{pxjahyper}
% ---------
\usepackage{otf} % ローマ数字
\usepackage{arydshln}% 表のダッシュを使う

% ローマ数字のコマンド化
\newcounter{num}
\newcommand{\Rnum}[1]{\setcounter{num}{#1} (\Roman{num})}
\newcommand{\rnum}[1]{\setcounter{num}{#1} (\roman{num})}

% ベクトルコマンド
\renewcommand{\v}[1]{\bm{#1}}

% サブセクションの数字
\renewcommand{\thesubsection}{\arabic{subsection}}

% 目次の深さ指定
%\setcounter{tocdepth}{1}


% 数式番号の変更
\makeatletter
\@addtoreset{equation}{section}
\def\theequation{\thesection.\arabic{equation}}% renewcommand でもOK 数式番号を1.3みたいな感じに、章.何番目 と変更
\makeatother

% -------------------
% 作図
\usepackage{tikz}
\usetikzlibrary{intersections, calc, arrows.meta}

% -------------------
% 定理環境付近
\usepackage{tcolorbox} % 色付きの囲み
\tcbuselibrary{breakable, skins, theorems}
\usepackage{ascmac} % 囲み \begin{itembox}ができる。

% ------ 新スタイル設定
\newtheoremstyle{mystyle}%    % スタイル名
{}%                       % 上部スペース
{}%                       % 下部スペース
{\normalfont}%           % 本文フォント
{}%                       % インデント量
{\bf}%                   % 見出しフォント
{}%                       % 見出し後の句読点,'.'
{\newline}%              % 見出し後のスペース, ' ' or \newline
{\underline{\thmname{#1}\thmnumber{#2}\thmnote{(#3)}}}%
%見出しの書式(can be left empty, meaning 'normal')
\theoremstyle{mystyle} % スタイルの適用

\newtheorem{theorem}{定理}[section]
\newtheorem{yousei}{要請}

\newtheorem{mondai}{小問}
\renewcommand{\themondai}{} % 番号なくすため

\newtheorem{kaitou}{解答}
\renewcommand{\thekaitou}{}

\newtheorem{teigi}{定義}
\renewcommand{\theteigi}{}
\newtheorem{lemma}{補題}[section]
\renewcommand{\proofname}{証明}

% ------------proof環境をいじる
\makeatletter %use at mark
\renewenvironment{proof}[1][\proofname]{\par
	\pushQED{\qed}%
	\normalfont \topsep6\p@\@plus6\p@\relax
	\trivlist
	\item[\hskip\labelsep
	            \itshape
	            {\bf\underline{#1}}]\ignorespaces
	% {\bf\underline{#1}\@addpunct{.}}]\ignorespaces % ピリオドあり
}{%
	\popQED\endtrivlist\@endperfalse
}
\makeatother %end at mark

% ----------

% プログラムコードの挿入
\usepackage{listings, jvlisting}

\definecolor{identifier}{rgb}{0.611, 0.862, 0.996}  % LightBlue (変数)
\definecolor{comment}{rgb}{0.415, 0.600, 0.333}     % DeepGreen (コメント)
\definecolor{keyword1}{rgb}{0.768, 0.521, 0.749}    % Purple    (予約語)
\definecolor{keyword2}{rgb}{0.862, 0.862, 0.666}    % Yellow    (関数名)
\definecolor{keyword3}{rgb}{0.2, 0.8, 1}            % Blue      (大域変数)
\definecolor{keyword4}{rgb}{0.854, 0.439, 0.839}    % Pink      (定数)
\definecolor{string}{rgb}{0.847, 0.560, 0.458}      % Orange    (文字列)

\lstdefinelanguage{myscilab}{
	morekeywords=[1]{   % 予約語 標準ライブラリを含む
			for, end, if, then, function, endfunction, 
			clear, strtod, read_csv, size, zeros, mean, sqrt, floor, string, disp, sum, 
			cov, cell, cat, exp, matrix, inv, det, int, gsort, tabul, max, 
		},
	morekeywords=[2]{   % 関数名
			get_class_range,
			bayes_train,
			bayes_valid,
		},
	morekeywords=[3]{   % 大域変数
			DATADIR, ALLDATA,
			FEATURE1, FEATURE2, FEATURES,
			FDIM, FDATA, ALLDATA,
			N_CLASS, N_FDATA, N_CLASS_DATA,
		},
	morekeywords=[4]{ \%inf, \%pi, },   % 定数
	sensitive=true,         % 大文字小文字の区別
	morecomment=[l]{//},    % コメントアウト
	morestring=[d]{'}{'},   % 文字列 (シングル)
	morestring=[b]{"},      % 文字列 (ダブル)
	alsoletter={\%},        % 文字としてに\%を追加
}

\lstset{
	language=myscilab,  % 新しく定義した言語を設定
	frame = tbrl,         % 枠設定
	breaklines=true,    % 行が長くなった場合自動改行
	breakindent=12pt,   % 自動改行時のインデント
	columns=fixed,      % 文字の間隔を統一
	basewidth=0.5em,    % 文字の横のサイズを小さく
	%numbers=left,       % 行数の位置
	numberstyle={\scriptsize \color{white}},  % 行数のフォント
	stepnumber=1,       % 行数の増間
	numbersep=1zw,      % 行数の余白
	xrightmargin=0zw,   % 左の余白
	xleftmargin=2zw,    % 右の余白
	framexleftmargin=18pt,  % フレームからの左の余白
	keepspaces=true,    % スペースを省略せず保持
	lineskip=-0.2ex,    % 枠線の途切れ防止
	tabsize = 4,        % タブ数
	showstringspaces=false,  %文字列中の半角スペースを表示させない
	keepspaces=true, % コード中のスペースを省略しない
	%%%%% VSCode風 の style & color %%%%%
	backgroundcolor={\color[gray]{0.1}},                    %背景色
	basicstyle     ={\small\ttfamily \color{white}},        % 基礎の文字のフォント設定
	identifierstyle={\small          \color{identifier}},   % 変数名などのフォント設定
	commentstyle   ={\small          \color{comment}},      % コメントのフォント設定
	keywordstyle   =[1]{\small\bfseries \color{keyword1}},  % 予約語のフォント設定
	keywordstyle   =[2]{\small\bfseries \color{keyword2}},  % 関数名のフォント設定
	keywordstyle   =[3]{\small\bfseries \color{keyword3}},  % 大域変数のフォント設定
	keywordstyle   =[4]{\small\bfseries \color{keyword4}},  % 定数のフォント設定
	stringstyle    ={\small\ttfamily    \color{string}},    % 文字列のフォント設定
	%%%%% VSCode風 の style & color %%%%%
}


% \lstset{
% 	%プログラム言語(複数の言語に対応,C,C++も可)
% 	language = {C},
% 	%背景色と透過度
% 	backgroundcolor={\color[gray]{.90}},
% 	%枠外に行った時の自動改行
% 	breaklines = true,
% 	%自動改行後のインデント量(デフォルトでは20[pt])
% 	breakindent = 10pt,
% 	%標準の書体
% 	basicstyle = {\ttfamily\scriptsize},
% 	%コメントの書体
% 	commentstyle = {\itshape \color[rgb]{0,0.5,0}},
% 	%関数名等の色の設定
% 	classoffset = 0,
% 	%キーワード(int, ifなど)の書体
% 	keywordstyle = {\bfseries \color[rgb]{1,0,0}},
% 	%""で囲まれたなどの"文字"の書体
% 	stringstyle = {\ttfamily \color[rgb]{0,0,1}},
% 	%枠 "t"は上に線を記載, "T"は上に二重線を記載
% 	%他オプション: leftline, topline, bottomline, lines, single, shadowbox
% 	frame = tbrl,
% 	%frameまでの間隔(行番号とプログラムの間)
% 	framesep = 5pt,
% 	%行番号の位置
% 	numbers = left,
% 	%行番号の間隔
% 	stepnumber = 1,
% 	%行番号の書体
% 	numberstyle = \tiny,
% 	%タブの大きさ
% 	tabsize = 4,
% 	%キャプションの場所("tb"ならば上下両方に記載)
% 	captionpos = t
% }


